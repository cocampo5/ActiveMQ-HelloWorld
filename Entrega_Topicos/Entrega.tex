\documentclass[]{article}
\usepackage[utf8]{inputenc}
\usepackage{graphicx} %Para insertar imágenes
\graphicspath{ {images/} }
\usepackage[spanish]{babel}
\usepackage{subfigure}
\usepackage[section]{placeins}
\usepackage{listings}
%\usepackage{multicol}
%\usepackage{minted}
%\usepackage{amssymb}% http://ctan.org/pkg/amssymb
\usepackage{pifont}% http://ctan.org/pkg/pifont
\usepackage{geometry}
\usepackage{helvet}
\usepackage{hyperref}
\renewcommand{\familydefault}{\sfdefault}
\geometry{
letterpaper,
total={215.9mm,279.4mm},
left=30mm,
right=30mm,
top=30mm,
bottom=30mm,
}
\title{Ejecución del HelloWorld de ActiveMQ}
\author{
  Gutierrez Gómez,Mateo\\
  \texttt{mgutie22@eafit.edu.co}
  \and
  Arango Carvajal,César\\
  \texttt{carang44@eafit.edu.co}
  \and
  Ocampo Quintero, Cristóbal\\
  \texttt{cocampo5@eafit.edu.co}
}
\begin{document}
\maketitle
\tableofcontents

\section{Introducción y pre requisitos}
Los requisitos mínimos para la ejecución del ejemplo son:
\begin{itemize}
\item Java 1.7 o mayor.
\item Apache ActiveMQ descargado y extraido.
\end{itemize}
\section{Compilación}
\begin{enumerate}
\item Compilar proyecto en NetBeans
Simplemente abrir el proyecto en NetBeans y luego click derecho sobre el proyecto y la opción 'Clean and Build'
\item Compilar manualmente los archivos fuentes
Para compilar los archivos ejecutar los siguientes comandos dentro de la carpeta "src" del proyecto, el proyecto se descarga desde el repositorio alojado en GitHub:
\lstset{language=bash} 
\begin{lstlisting}
$ git clone https://github.com/cocampo5/ActiveMQ-HelloWorld.git
\end{lstlisting}
Luego dentro la estructura del proyecto se busca la carpeta 'src' y se ejecuta para cada archivo .java el siguiente código
\begin{lstlisting}
$ javac -cp [path_al.jar_de_activemq]:. Clase.java
\end{lstlisting}
\end{enumerate}
\section{Ejecución}
Primero que todo en una consola entrar al directorio bin del ActiveMQ y ejecutar en Linux:
\begin{lstlisting}
$ sudo .\activemq console 
\end{lstlisting}
o Windows
\begin{lstlisting}
$ activemq start
\end{lstlisting}
La ejecución depende de si la compilación fue realizada en NetBeans o manual
\begin{enumerate}
\item Compilado por NetBeans
Si el proyecto se compiló directamente entrar a build/classes (deben haber cinco .class).
\item Compilado manual
Si el proyecto se compiló manualmente entrar a la carpeta src (donde ahora deben haber además de los .java otros cinco .class).
\end{enumerate}
Ya estando en las respectivas carpeta se ejecutará en dos consolas diferentes:
\begin{itemize}
\item Para Productor/Consumidor con Colas
\begin{lstlisting}
java -cp [path_al.jar_de_activemq]:. EjemploConsumidor
java -cp [path_al.jar_de_activemq]:. EjemploProductor
\end{lstlisting}
\item Para Productor/Consumidor con Tópicos
\begin{lstlisting}
java -cp [path_al.jar_de_activemq]:. TopicListener
java -cp [path_al.jar_de_activemq]:. TopicPublisher
\end{lstlisting}
\end{itemize}
Nota: la ejecución es prácticamente igual en el servidor.
\end{document}
